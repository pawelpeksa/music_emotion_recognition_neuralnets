\section{Wnioski}\label{rozdzial_wnioski}
\subsection{Ocena działania systemu}
System do pewnego stopnia spełnił postawione przed nim zadania. Otrzymane wyniki pokazują, że stworzona sieć neuronowa nie modeluje w~znaczącym stopniu zależności pomiędzy cechami dźwiękowymi utworów muzycznych, a~parametrami określającymi nastrój tych utworów. Współczynniki przedstawione w~tabeli \ref{table:coeff} wskazują na słabą korelację pomiędzy odpowiedziami sieci, a~oczekiwanymi wartościami. Można jednak zaobserwować pewną korelację pomiędzy uzyskaną przez system oceną nastroju, a~oceną badanych. Wskazuje to, że możliwe jest zbudowanie efektywnego systemu klasyfikującego. Po lekturze rozdziału \ref{rozdzial_zaleznosc} zauważamy, że jedynie 5 z 11 analizowanych cech daje podstawy dla sieci pozwalające wyznaczyć wartości pobudzenia oraz zadowolenia, co nie wątpliwie jest jedną z przyczyn niskich współczynników korelacji. Należy jednak zauważyć, że wszystkie utwory były analizowane całościowo tzn. sygnał nie był dzielony na mniejsze fragmenty, co prawdopodobnie jest przyczyną niezadowalających wyników. Nastrój muzyki zmienia się w czasie, co także powinno być wzięte pod uwagę w przypadku realizacji zadania rozpoznawania emocji reprezentowanych przez muzykę, co zostało zrobione w~innych pracach podejmujących podobną tematykę\cite{musicANN1}\cite{musicANN2} i~dało wyraźnie lepsze rezultaty. Należy jednak mieć na uwadze jednak także bazę danych na której przeprowadzono badania, gdyż duże znaczenie może mieć także jej zróżnicowanie pod kątem gatunków muzycznych. 
\subsection{Propozycja usprawnienia}
Biorąc pod uwagę wspominane prawdopodobne przyczyny niskiej efektywności systemu, pierwszą, najważniejszą propozycją usprawnienia jest dzielenie sygnału audio na mniejsze fragmenty, ocena ich oraz uśrednienie wyników, co mogłoby dać lepsze wyniki. Należałoby wtedy także przekonać się czy cechy dźwięku wykorzystane w programie wciąż nie mają wpływu na wartości parametrów zadowolenia oraz pobudzenia, a~także rozważyć użycie innych cech, co nie stanowiłoby wielkiego wyzwania dzięki bibliotece programistycznej Essentia, gdyż oferuje ona kilkadziesiąt możliwych do wykorzystania algorytmów. Oprócz cech wynikających z~samego sygnału audio, możliwe jest także analiza tekstu utworów, który również ma znaczący wpływ na nastrój utworu. Dodatkową sugestią w kwestii rozpoznawania nastroju mogą być także okładki płyt z których pochodzą utwory muzyczne, co również mogłoby usprawnić ocenę. Innym kierunkiem rozwoju aplikacji prezentowanej w pracy jest także ocena utworów muzycznych pod kątem ich gatunków.